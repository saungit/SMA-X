%%%%%%%%%%%%%%%%%%%%%%%%%%%%%%%%%%%%%%%%%
% The Legrand Orange Book
% LaTeX Template
% Version 2.2 (30/3/17)
%
% This template has been downloaded from:
% http://www.LaTeXTemplates.com
%
% Original author:
% Mathias Legrand (legrand.mathias@gmail.com) with modifications by:
% Vel (vel@latextemplates.com)
%
% License:
% CC BY-NC-SA 3.0 (http://creativecommons.org/licenses/by-nc-sa/3.0/)
%
% Compiling this template:
% This template uses biber for its bibliography and makeindex for its index.
% When you first open the template, compile it from the command line with the 
% commands below to make sure your LaTeX distribution is configured correctly:
%
% 1) pdflatex main
% 2) makeindex main.idx -s StyleInd.ist
% 3) biber main
% 4) pdflatex main x 2
%
% After this, when you wish to update the bibliography/index use the appropriate
% command above and make sure to compile with pdflatex several times 
% afterwards to propagate your changes to the document.
%
% This template also uses a number of packages which may need to be
% updated to the newest versions for the template to compile. It is strongly
% recommended you update your LaTeX distribution if you have any
% compilation errors.
%
% Important note:
% Chapter heading images should have a 2:1 width:height ratio,
% e.g. 920px width and 460px height.
%
%%%%%%%%%%%%%%%%%%%%%%%%%%%%%%%%%%%%%%%%%

%----------------------------------------------------------------------------------------
%	PACKAGES AND OTHER DOCUMENT CONFIGURATIONS
%----------------------------------------------------------------------------------------

\documentclass[11pt,fleqn]{book} % Default font size and left-justified equations



%----------------------------------------------------------------------------------------

\input{structure} % Insert the commands.tex file which contains the majority of the structure behind the template

\begin{document}

%----------------------------------------------------------------------------------------
%	TITLE PAGE
%----------------------------------------------------------------------------------------

\begingroup
\thispagestyle{empty}
\begin{tikzpicture}[remember picture,overlay]
\node[inner sep=0pt] (background) at (current page.center) {\includegraphics[width=\paperwidth]{background}};
\draw (current page.center) node [fill=ocre!30!white,fill opacity=0.6,text opacity=1,inner sep=1cm]{\Huge\centering\bfseries\sffamily\parbox[c][][t]{\paperwidth}{\centering The Search for a Title\\[15pt] % Book title
{\Large A Profound Subtitle}\\[20pt] % Subtitle
{\huge Dr. John Smith}}}; % Author name
\end{tikzpicture}
\vfill
\endgroup

%----------------------------------------------------------------------------------------
%	COPYRIGHT PAGE
%----------------------------------------------------------------------------------------

\newpage
~\vfill
\thispagestyle{empty}

\noindent Copyright \copyright\ 2013 John Smith\\ % Copyright notice

\noindent \textsc{Published by Publisher}\\ % Publisher

\noindent \textsc{book-website.com}\\ % URL

\noindent Licensed under the Creative Commons Attribution-NonCommercial 3.0 Unported License (the ``License''). You may not use this file except in compliance with the License. You may obtain a copy of the License at \url{http://creativecommons.org/licenses/by-nc/3.0}. Unless required by applicable law or agreed to in writing, software distributed under the License is distributed on an \textsc{``as is'' basis, without warranties or conditions of any kind}, either express or implied. See the License for the specific language governing permissions and limitations under the License.\\ % License information

\noindent \textit{First printing, March 2013} % Printing/edition date

%----------------------------------------------------------------------------------------
%	TABLE OF CONTENTS
%----------------------------------------------------------------------------------------

%\usechapterimagefalse % If you don't want to include a chapter image, use this to toggle images off - it can be enabled later with \usechapterimagetrue

\chapterimage{chapter_head_1.pdf} % Table of contents heading image

\pagestyle{empty} % No headers

\tableofcontents % Print the table of contents itself

\cleardoublepage % Forces the first chapter to start on an odd page so it's on the right

\pagestyle{fancy} % Print headers again

%----------------------------------------------------------------------------------------
%	PART
%----------------------------------------------------------------------------------------

\part{Part One}

%----------------------------------------------------------------------------------------
%	CHAPTER 1
%----------------------------------------------------------------------------------------

\chapterimage{chapter_head_2.pdf} % Chapter heading image

\chapter{Persamaan dan Pertidaksamaan Linear Satu Variabel yang Memuat Nilai Mutlak}

\section{Paragraphs of Text}\index{Paragraphs of Text}

Lorem ipsum dolor sit amet, consectetuer adipiscing elit. Ut purus elit, vestibulum ut, placerat ac, adipiscing vitae, felis. Curabitur dictum gravida mauris. Nam arcu libero, nonummy eget, consectetuer id, vulputate a, magna. Donec vehicula augue eu neque. Pellentesque habitant morbi tristique senectus et netus et malesuada fames ac turpis egestas. Mauris ut leo. Cras viverra metus rhoncus sem. Nulla et lectus vestibulum urna fringilla ultrices. Phasellus eu tellus sit amet tortor gravida placerat. Integer sapien est, iaculis in, pretium quis, viverra ac, nunc. Praesent eget sem vel leo ultrices bibendum. Aenean faucibus. Morbi dolor nulla, malesuada eu, pulvinar at, mollis ac, nulla. Curabitur auctor semper nulla. Donec varius orci eget risus. Duis nibh mi, congue eu, accumsan eleifend, sagittis quis, diam. Duis eget orci sit amet orci dignissim rutrum.

Nam dui ligula, fringilla a, euismod sodales, sollicitudin vel, wisi. Morbi auctor lorem non justo. Nam lacus libero, pretium at, lobortis vitae, ultricies et, tellus. Donec aliquet, tortor sed accumsan bibendum, erat ligula aliquet magna, vitae ornare odio metus a mi. Morbi ac orci et nisl hendrerit mollis. Suspendisse ut massa. Cras nec ante. Pellentesque a nulla. Cum sociis natoque penatibus et magnis dis parturient montes, nascetur ridiculus mus. Aliquam tincidunt urna. Nulla ullamcorper vestibulum turpis. Pellentesque cursus luctus mauris.

Nulla malesuada porttitor diam. Donec felis erat, congue non, volutpat at, tincidunt tristique, libero. Vivamus viverra fermentum felis. Donec nonummy pellentesque ante. Phasellus adipiscing semper elit. Proin fermentum massa ac quam. Sed diam turpis, molestie vitae, placerat a, molestie nec, leo. Maecenas lacinia. Nam ipsum ligula, eleifend at, accumsan nec, suscipit a, ipsum. Morbi blandit ligula feugiat magna. Nunc eleifend consequat lorem. Sed lacinia nulla vitae enim. Pellentesque tincidunt purus vel magna. Integer non enim. Praesent euismod nunc eu purus. Donec bibendum quam in tellus. Nullam cursus pulvinar lectus. Donec et mi. Nam vulputate metus eu enim. Vestibulum pellentesque felis eu massa.

Quisque ullamcorper placerat ipsum. Cras nibh. Morbi vel justo vitae lacus tincidunt ultrices. Lorem ipsum dolor sit amet, consectetuer adipiscing elit. In hac habitasse platea dictumst. Integer tempus convallis augue. Etiam facilisis. Nunc elementum fermentum wisi. Aenean placerat. Ut imperdiet, enim sed gravida sollicitudin, felis odio placerat quam, ac pulvinar elit purus eget enim.

Nam dui ligula, fringilla a, euismod sodales, sollicitudin vel, wisi. Morbi auctor lorem non justo. Nam lacus libero, pretium at, lobortis vitae, ultricies et, tellus. Donec aliquet, tortor sed accumsan bibendum, erat ligula aliquet magna, vitae ornare odio metus a mi. Morbi ac orci et nisl hendrerit mollis. Suspendisse ut massa. Cras nec ante. Pellentesque a nulla. Cum sociis natoque penatibus et magnis dis parturient montes, nascetur ridiculus mus. Aliquam tincidunt urna. Nulla ullamcorper vestibulum turpis. Pellentesque cursus luctus mauris.

Nulla malesuada porttitor diam. Donec felis erat, congue non, volutpat at, tincidunt tristique, libero. Vivamus viverra fermentum felis. Donec nonummy pellentesque ante. Phasellus adipiscing semper elit. Proin fermentum massa ac quam. Sed diam turpis, molestie vitae, placerat a, molestie nec, leo. Maecenas lacinia. Nam ipsum ligula, eleifend at, accumsan nec, suscipit a, ipsum. Morbi blandit ligula feugiat magna. Nunc eleifend consequat lorem. Sed lacinia nulla vitae enim. Pellentesque tincidunt purus vel magna. Integer non enim. Praesent euismod nunc eu purus. Donec bibendum quam in tellus. Nullam cursus pulvinar lectus. Donec et mi. Nam vulputate metus eu enim. Vestibulum pellentesque felis eu massa.

Quisque ullamcorper placerat ipsum. Cras nibh. Morbi vel justo vitae lacus tincidunt ultrices. Lorem ipsum dolor sit amet, consectetuer adipiscing elit. In hac habitasse platea dictumst. Integer tempus convallis augue. Etiam facilisis. Nunc elementum fermentum wisi. Aenean placerat. Ut imperdiet, enim sed gravida sollicitudin, felis odio placerat quam, ac pulvinar elit purus eget enim.

%------------------------------------------------

\section{Citation}\index{Citation}

This statement requires citation \cite{book_key}; this one is more specific \cite[122]{article_key}.

%------------------------------------------------

\section{Lists}\index{Lists}

Lists are useful to present information in a concise and/or ordered way\footnote{Footnote example...}.

\subsection{Numbered List}\index{Lists!Numbered List}

\begin{enumerate}
\item The first item
\item The second item
\item The third item
\end{enumerate}

\subsection{Bullet Points}\index{Lists!Bullet Points}

\begin{itemize}
\item The first item
\item The second item
\item The third item
\end{itemize}

\subsection{Descriptions and Definitions}\index{Lists!Descriptions and Definitions}

\begin{description}
\item[Name] Description
\item[Word] Definition
\item[Comment] Elaboration
\end{description}

%----------------------------------------------------------------------------------------
%	CHAPTER 2
%----------------------------------------------------------------------------------------

\chapter{Linear Tiga Variabel}

\section{Theorems}\index{Theorems}

This is an example of theorems.

\subsection{Several equations}\index{Theorems!Several Equations}
This is a theorem consisting of several equations.

\begin{theorem}[Name of the theorem]
In $E=\mathbb{R}^n$ all norms are equivalent. It has the properties:
\begin{align}
& \big| ||\mathbf{x}|| - ||\mathbf{y}|| \big|\leq || \mathbf{x}- \mathbf{y}||\\
&  ||\sum_{i=1}^n\mathbf{x}_i||\leq \sum_{i=1}^n||\mathbf{x}_i||\quad\text{where $n$ is a finite integer}
\end{align}
\end{theorem}

\subsection{Single Line}\index{Theorems!Single Line}
This is a theorem consisting of just one line.

\begin{theorem}
A set $\mathcal{D}(G)$ in dense in $L^2(G)$, $|\cdot|_0$. 
\end{theorem}

%------------------------------------------------

\section{Definitions}\index{Definitions}

This is an example of a definition. A definition could be mathematical or it could define a concept.

\begin{definition}[Definition name]
Given a vector space $E$, a norm on $E$ is an application, denoted $||\cdot||$, $E$ in $\mathbb{R}^+=[0,+\infty[$ such that:
\begin{align}
& ||\mathbf{x}||=0\ \Rightarrow\ \mathbf{x}=\mathbf{0}\\
& ||\lambda \mathbf{x}||=|\lambda|\cdot ||\mathbf{x}||\\
& ||\mathbf{x}+\mathbf{y}||\leq ||\mathbf{x}||+||\mathbf{y}||
\end{align}
\end{definition}

%------------------------------------------------

\section{Notations}\index{Notations}

\begin{notation}
Given an open subset $G$ of $\mathbb{R}^n$, the set of functions $\varphi$ are:
\begin{enumerate}
\item Bounded support $G$;
\item Infinitely differentiable;
\end{enumerate}
a vector space is denoted by $\mathcal{D}(G)$. 
\end{notation}

%------------------------------------------------

\section{Remarks}\index{Remarks}

This is an example of a remark.

\begin{remark}
The concepts presented here are now in conventional employment in mathematics. Vector spaces are taken over the field $\mathbb{K}=\mathbb{R}$, however, established properties are easily extended to $\mathbb{K}=\mathbb{C}$.
\end{remark}

%------------------------------------------------

\section{Corollaries}\index{Corollaries}

This is an example of a corollary.

\begin{corollary}[Corollary name]
The concepts presented here are now in conventional employment in mathematics. Vector spaces are taken over the field $\mathbb{K}=\mathbb{R}$, however, established properties are easily extended to $\mathbb{K}=\mathbb{C}$.
\end{corollary}

%------------------------------------------------

\section{Propositions}\index{Propositions}

This is an example of propositions.

\subsection{Several equations}\index{Propositions!Several Equations}

\begin{proposition}[Proposition name]
It has the properties:
\begin{align}
& \big| ||\mathbf{x}|| - ||\mathbf{y}|| \big|\leq || \mathbf{x}- \mathbf{y}||\\
&  ||\sum_{i=1}^n\mathbf{x}_i||\leq \sum_{i=1}^n||\mathbf{x}_i||\quad\text{where $n$ is a finite integer}
\end{align}
\end{proposition}

\subsection{Single Line}\index{Propositions!Single Line}

\begin{proposition} 
Let $f,g\in L^2(G)$; if $\forall \varphi\in\mathcal{D}(G)$, $(f,\varphi)_0=(g,\varphi)_0$ then $f = g$. 
\end{proposition}

%------------------------------------------------

\section{Examples}\index{Examples}

This is an example of examples.

\subsection{Equation and Text}\index{Examples!Equation and Text}

\begin{example}
Let $G=\{x\in\mathbb{R}^2:|x|<3\}$ and denoted by: $x^0=(1,1)$; consider the function:
\begin{equation}
f(x)=\left\{\begin{aligned} & \mathrm{e}^{|x|} & & \text{si $|x-x^0|\leq 1/2$}\\
& 0 & & \text{si $|x-x^0|> 1/2$}\end{aligned}\right.
\end{equation}
The function $f$ has bounded support, we can take $A=\{x\in\mathbb{R}^2:|x-x^0|\leq 1/2+\epsilon\}$ for all $\epsilon\in\intoo{0}{5/2-\sqrt{2}}$.
\end{example}

\subsection{Paragraph of Text}\index{Examples!Paragraph of Text}

\begin{example}[Example name]
\lipsum[2]
\end{example}

%------------------------------------------------

\section{Exercises}\index{Exercises}

This is an example of an exercise.

\begin{exercise}
This is a good place to ask a question to test learning progress or further cement ideas into students' minds.
\end{exercise}

%------------------------------------------------

\section{Problems}\index{Problems}

\begin{problem}
What is the average airspeed velocity of an unladen swallow?
\end{problem}

%------------------------------------------------

\section{Vocabulary}\index{Vocabulary}

Define a word to improve a students' vocabulary.

\begin{vocabulary}[Word]
Definition of word.
\end{vocabulary}

%----------------------------------------------------------------------------------------
%	PART
%----------------------------------------------------------------------------------------

\part{Part Two}

%----------------------------------------------------------------------------------------
%	CHAPTER 3
%----------------------------------------------------------------------------------------

\chapterimage{chapter_head_1.pdf} % Chapter heading image

\chapter{Fungsi}

\section{Table}\index{Table}

\begin{table}[h]
\centering
\begin{tabular}{l l l}
\toprule
\textbf{Treatments} & \textbf{Response 1} & \textbf{Response 2}\\
\midrule
Treatment 1 & 0.0003262 & 0.562 \\
Treatment 2 & 0.0015681 & 0.910 \\
Treatment 3 & 0.0009271 & 0.296 \\
\bottomrule
\end{tabular}
\caption{Table caption}
\end{table}

%------------------------------------------------

%----------------------------------------------------------------------------------------
%	CHAPTER 4
%----------------------------------------------------------------------------------------

\chapterimage{chapter_head_1.pdf} % Chapter heading image

\chapter{Trigonometri}

\section{Pengukuran Sudut}\index{Pengukuran Sudut}

%------------------------------------------------
\section{Perbandingan Trigonometri pada Segitiga Siku-Siku}\index{Perbandingan Trigonometri pada Segitiga Siku-Siku}

%------------------------------------------------
\section{Sudut-sudut Berelasi}\index{Sudut-sudut Berelasi}

%------------------------------------------------

\section{Identitas Trigonometri}\index{Identitas Trigonometri}

%------------------------------------------------
\section{Aturan Sinus dan Cosinus}\index{Aturan Sinus dan Cosinus}

%------------------------------------------------
\section{Fungsi Trigonometri}\index{Fungsi Trigonometri}

Pada subbab ini, kita akan  mengkaji bagaimana konsep trigonometri jika dipandang sebagai suatu fungsi. Mengingat kembali konsep fungsi pada Bab 3, fungsi $f(x)$ harus terdefinisi pada daerah asalnya. Jika $y = f(x) = \sin x$, maka daerah asalnya adalah semua x bilangan real. Namun, mengingat satuan sudut  (subbab 4.1) dan nilai-nilai perbandingan trigonometri (yang disajikan pada Tabel 4.3),  pada kesempatan ini, kita hanya mengkaji  untuk  ukuran sudut dalam derajat. Mari kita sketsakan grafik fungsi $y = f(x) = \sin x$,untuk $0 \leq x\leq 2\pi$.\\

\begin{enumerate}
\item \textbf{Grafik Fungsi $y = \sin x$, dan $y = \cos x$ untuk $0 \leq x\leq 2\pi$}\\
\begin{problem}
Dengan keterampilan kamu dalam menggambar suatu fungsi (Bab 3), gambarkan grafik fungsi $y = \sin x$, untuk $0 \leq x\leq 2\pi$.\\
\end{problem}
\textbf{Alternatif Penyelesaian}\\
Dengan mencermati nilai-nilai sinus untuk semua sudut istimewa yang disajikan pada Tabel 4.3, kita dapat memasangkan ukuran sudut dengan nilai sinus untuk setiap sudut tersebut, sebagai berikut.\\

\includegraphics[scale=0.75]{fungsi_trigonometri_1}

Selanjutnya pada koordinat kartesius, kita menempatkan pasangan titiktitik untuk menemukan suatu kurva  yang melalui semua pasangan titik-titik tersebut. Selengkapnya disajikan pada Gambar berikut ini.\\

\begin{figure}[!ht]
\begin{center}
\includegraphics[scale=0.75]{grafik_fungsi_trigonometri_1}
\caption{Grafik fungsi $y = \sin x$, untuk $0 \leq x\leq 2\pi$}
\end{center}
\end{figure}

Dari grafik di atas,  kita dapat merangkum beberapa data dan informasi seperti berikut.
\begin{itemize}
\item Untuk semua ukuran sudut $x$,  nilai maksimum fungsi $y = \sin x$ adalah 1, dan nilai minimumnya adalah -1.
\item Kurva fungsi $y = \sin x$, berupa gelombang.
\item Untuk 1 periode (1 putaran penuh) kurva fungsi $y = \sin x$, memiliki 1 gunung dan 1 lembah.
\item Nilai fungsi sinus berulang saat berada pada lembah atau gunung yang sama.
\item Untuk semua ukuran sudut $x$, daerah hasil fungsi $y = \sin x$, adalah $1 \leq y \leq 1$. Dengan konsep grafik fungsi $y = \sin x$, dapat dibentuk kombinasi fungsi sinus.
\end{itemize}

Misalnya $y = 2.\sin x$, $y = \sin 2x$, dan $y = \sin (x+\pi/2)$ . Selengkapnya dikaji pada contoh berikut.

\begin{example}
Gambarkan grafik fungsi $y = \sin 2x$ dan $y = \sin (x+\pi/2)$, untuk $0 \leq x\leq 2\pi$. Kemudian tuliskanlah perbedaan kedua grafik tersebut.\\

\textbf{Alternatif Penyelesaian}\\
Dengan menggunakan nilai-nilai perbandingan trigonometri yang disajikan pada Tabel 4.3, maka pasangan titik-titik untuk fungsi $y = \sin 2x$, untuk $0 \leq x\leq 2\pi$ adalah:\\
Untuk $x = 0$, maka nilai fungsi adalah $y = \sin 2.(0) = \sin 0 = 0 \Rightarrow (0, 0)$\\
Untuk $x = (\pi/6)$, maka nilai fungsi adalah $y = \sin 2. (\pi/6) = \sin \pi/3 = \sqrt{3}/2 \Rightarrow(\pi/6,\sqrt{3}/2)$\\
Untuk $x = \pi/4$, maka nilai fungsi adalah $y = \sin 2. (\pi/4) = \sin \pi/2 = 1 \Rightarrow(\pi/4,1)$.\\
Demikian seterusnya hingga\\
untuk $x = 2\pi$, maka niali fungsi adalah $y = \sin 2.(2\pi) = \sin 4\pi = \sin 0 = 0 \Rightarrow (2\pi, 0)$\\
Selengkapnya pasangan titik-titik untuk fungsi $y = \sin 2x$, $0 \leq x\leq 2\pi$, yaitu

\includegraphics[scale=0.75]{fungsi_trigonometri_2}

Dengan  pasangan titik-titik tersebut, maka grafik fungsi $y = \sin 2x$, $0 \leq x\leq 2\pi$ disajikan pada Gambar.\\

\begin{figure}[!ht]
\begin{center}
\includegraphics[scale=0.75]{grafik_fungsi_trigonometri_2} 
\caption{Grafik fungsi $y = \sin 2x$, untuk $0 \leq x\leq 2\pi$}
\end{center}
\end{figure}

Berbeda dengan fungsi $y = \sin 2x$, setiap besar sudut dikalikan dua, tetapi untuk fungsi $y = \sin(x+\pi/2)$, setiap besar sudut ditambah $\pi/2$ atau $90^o$.\\
Sekarang kita akan menggambarkan fungsi $y = \sin(x+\pi/2)$, untuk $0 \leq x\leq 2\pi$.\\
Coba kita perhatikan kembali, bahwa $\sin(x+\pi/2) = \cos x$. Artinya, sekarang kita akan menggambarkan fungsi $y = \cos x$, untuk $0 \leq x\leq 2\pi$. Dengan menggunakan nilai-nilai cosinus yang diberikan pada Tabel kita dapat merangkumkan pasangan titik-titik  yang memenuhi fungsi $y = \cos x$, untuk $0 \leq x\leq 2\pi$, sebagai berikut.

\includegraphics[scale=0.75]{fungsi_trigonometri_3}

Dengan demikian, grafik fungsi $y = \cos x$, untuk $0 \leq x\leq 2\pi$, disajikan pada Gambar berikut.\\

\begin{figure}[!ht]
\begin{center}
\includegraphics[scale=0.75]{grafik_fungsi_trigonometri_3} 
\caption{Grafik fungsi $y = \cos x$, untuk $0 \leq x\leq 2\pi$}
\end{center}
\end{figure}

Dari kajian grafik, grafik fungsi $y = \sin 2x$ sangat berbeda dengan grafik fungsi $y = \sin (x+\pi/2)  = \cos x$, meskipun untuk domain yang sama. Grafik $y = \sin 2x$, memiliki 2 gunung dan 2 lembah, sedangkan grafik fungsi $y = \sin (x+\pi/2)= \cos x$, hanya memiliki 1 lembah dan dua bagian setengah gunung. Nilai maksimum dan minimum fungsi $y = \sin 2x$ sama $y = \sin (x+\pi/2) = \cos x$ untuk domain yang sama. Selain itu, secara periodik, nilai fungsi $y = \sin 2x$ dan $y = \sin (x+\pi/2) = \cos x$, berulang, terkadang menaik dan terkadang menurun.
\end{example}
\begin{exercise}
Dengan pengetahuan dan keterampilan kamu akan tiga grafik di atas dan konsep yang sudah kamu miliki pada kajian fungsi, sekarang gambarkan dan gabungkan grafik $y = \sin x$ dan $y = \cos x$, untuk domain $0 \leq x\leq 2\pi$.\\
Rangkumkan hasil analisis yang kamu temukan atas grafik tersebut.
\end{exercise}
\item \textbf{Grafik Fungsi $y = tan x$, dan $y = \cos x$ untuk $0 \leq x\leq 2\pi$}\\
Kajian kita selanjutnya adalah untuk  menggambarkan grafik fungsi $y = \tan x$, untuk $0 \leq x\leq 2\pi$. Mari kita kaji grafik fungsi $y = \tan x$, melalui masalah berikut\\
\begin{problem}
Untuk domain $0 \leq x\leq 2\pi$, gambarkan grafik fungsi $y = \tan x$.\\
\end{problem}
\textbf{Alternatif Penyelesaian}\\
Dengan nilai-nilai tangen yang telah kita temukan pada Tabel 4.3 dan dengan pengetahuan serta keterampilan yang telah kamu pelajari tentang menggambarkan grafik suatu fungsi, kita dengan mudah memahami pasangan titik-titik berikut.\\

\includegraphics[scale=0.75]{fungsi_trigonometri_4}

Dengan demikian, grafik fungsi $y = \tan x$, untuk $0 \leq x\leq 2\pi$, seperti pada Gambar berikut ini.\\

\begin{figure}[!ht]
\begin{center}
\includegraphics[scale=0.75]{grafik_fungsi_trigonometri_4} 
\caption{Grafik fungsi $y = \tan x$, untuk $0 \leq x\leq 2\pi$}
\end{center}
\end{figure}

Dari grafik di atas, jelas kita lihat bahwa jika $x$ semakin mendekati $\pi/2$ (dari kiri), nilai fungsi semakin besar, tetapi tidak dapat ditentukan nilai terbesarnya. Sebaliknya, jika $x$ atau mendekati $\pi/2$ (dari kanan), maka nilai fungsi semakin kecil, tetapi tidak dapat ditentukan nilai terkecilnya. Kondisi ini berulang pada saat $x$ mendekati $3\pi/2$. Artinya, fungsi $y = \tan x$, tidak memiliki nilai maksimum dan minimum.
\end{enumerate}
%------------------------------------------------
%----------------------------------------------------------------------------------------
%	BIBLIOGRAPHY
%----------------------------------------------------------------------------------------

\chapter*{Bibliography}
\addcontentsline{toc}{chapter}{\textcolor{ocre}{Bibliography}}
\section*{Books}
\addcontentsline{toc}{section}{Books}
\printbibliography[heading=bibempty,type=book]
\section*{Articles}
\addcontentsline{toc}{section}{Articles}
\printbibliography[heading=bibempty,type=article]

%----------------------------------------------------------------------------------------
%	INDEX
%----------------------------------------------------------------------------------------

\cleardoublepage
\phantomsection
\setlength{\columnsep}{0.75cm}
\addcontentsline{toc}{chapter}{\textcolor{ocre}{Index}}
\printindex

%----------------------------------------------------------------------------------------

\end{document}
